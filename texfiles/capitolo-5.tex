\section{Prerequisiti e installazione}

Come prerequisito è necessario installare delle librerie che permettono di catturare pacchetti: gcc, flex, bison, zlib, libpcap, pcre, libdnet, tcpdump e libnghttp2.

Poi viene installato Snort con il package manager della distribuzione utilizzata.

Snort su CentOS viene installato nella directory \texttt{/usr/local/bin/snort}, è buona norma creare un link simbolico nel path \texttt{/usr/sbin/snort}. \texttt{/usr/sbin} è dedicata a programmi di gestione del sistema (normalmente non utilizzati dagli utenti ordinari).

\begin{verbatim}
    ln -s /usr/local/bin/snort /usr/sbin/snort  
\end{verbatim}

Per eseguire Snort in modo sicuro senza che esegua con i privilegi di root, bisogna creare un nuovo utente e un nuovo gruppo di utenti.

\begin{verbatim}
groupadd snort
useradd snort -r -s /sbin/nologin -c SNORT_IDS -g snort
\end{verbatim}

\section{Configurazione}

Per adeguare Snort alle esigenze della rete bisogna:

\begin{itemize}
    \item  modificare il file di configurazione.
    \item importare le regole.
    \item gestire i log.
\end{itemize}

Le principali directory utilizzate da Snort sono:

\texttt{/etc/snort}: in questa directory si trova il file di configurazione snort.conf e la sottodirectory che contiene le rules.

\texttt{/var/log/snort}: in questa directory vengono memorizzati i log, si possono specificare anche altri path, questo è quello di default.

Bisogna assicurarsi che lo user snort e il gruppo snort abbiano i permessi di lettura/ scrittura/ esecuzione.
Viene creato un file per le local.rules, ossia per tutte quelle regole customizzate dall'amministratore.

Ora si arriva al cuore della configurazione. Bisogna modificare il file snort.conf in modo da soddisfare i requisiti della rete.
Snort deve catturare e analizzare i pacchetti provenienti dall'interfaccia esterna della vm firewall. Per fare questo bisogna definire la variabile HOME\_NET e assegnargli il range di IP da proteggere 10.222.111.0/24. Viene creata anche una seconda variabile EXTERNAL\_NET per indicare la rete esterna, in questo caso rappresentante tutti gli indirizzi IP non compresi nella HOME\_NET

\begin{verbatim}
ipvar HOME_NET 10.222.111.0/24
ipvar EXTERNAL_NET !$HOME_NET
\end{verbatim}

Qui viene specificato il formato dei log, in questo caso utilizzando unified2 e tcpdump

\begin{verbatim}
# unified2
output unified2: filename merged.log, limit 128, nostamp
# pcap
output log_tcpdump: tcpdump.log
\end{verbatim}

Per includere le local.rules:

\begin{verbatim}
include /etc/snort/rules/local.rules
\end{verbatim}

Manca un ultimo passaggio: includere le community rules.

\subsection{Community Rules}


È possibile trovare le Community Rules sul sito ufficiale di Snort.


Per estrarre le regole e copiarle nella directory dove si trovano le rules:

\begin{verbatim}
tar -xvf ~/community.tar.gz -C ~/
cp ~/community-rules/* /etc/snort/rules
\end{verbatim}

Per includere tutte le regole della community in \texttt{/etc/snort/snort.conf}

\begin{verbatim}
include $RULE_PATH/snort.rules
\end{verbatim}

\section{Pulled Pork}

Ora è possibile procedere con l'installazione di PulledPork, il plugin che permette di scaricare automaticamente le nuove regole della community.

Viene installato PulledPork con il package manager della distribuzione utilizzata.

Lo script si trova nella directory \texttt{/usr/local/bin/} e viene reso eseguibile con il comando:

\begin{verbatim}
chmod +x /usr/local/bin/pulledpork.pl
\end{verbatim}

Per verificare il corretto funzionamento:

\begin{verbatim}
/usr/local/bin/pulledpork.pl –V
\end{verbatim}

Il file di configurazione si trova in \texttt{/etc/snort}, si chiama pulledpork.conf e va modificato.

Viene specificato il path delle community rules da aggiornare:

\begin{verbatim}
rule_path=/usr/local/etc/snort/rules/snort.rules
\end{verbatim}

Pulled pork richiede anche il path delle local.rules per costruire dei file sid-msg.map che contengono informazioni sui metadati (msg) delle local.rules

\begin{verbatim}
local_rules=/usr/local/etc/snort/rules/local.rules
\end{verbatim}

Pulled pork aggiorna una blocklist di indirizzi IP bloccati dalla comunità:
\begin{verbatim}
block_list=/usr/local/etc/snort/rules/iplists/default.blocklist
\end{verbatim}

\section{Il demone snortd}

Per far sì che Snort lavori in background, viene eseguito come demone. Un sistema di difesa ha l'esigenza di catturare pacchetti e analizzare il traffico durante tutto il periodo di attività della macchina. Per questo viene largamente sfruttata questa possibilità.

\begin{verbatim}
systemctl daemon-reload
systemctl start snortd
\end{verbatim}

Per avere la conferma che l'avvio del demone sia andata a buon fine:

\begin{verbatim}
systemctl status snortd
\end{verbatim}

\section{FwSnort}

FwSnort è uno script che traduce le regole di Snort in regole iptables. Alcune non hanno una traduzione diretta, tuttavia circa il 65\% delle regole possono essere tradotte con successo utilizzando l'iptables string match module (modulo che confronta l'inizio della stringa che compone la regola con la chiave di una mappa. Se viene trovata una corrispondenza restituisce l'oggetto corrispondete, un iptables). FwSnort analizza anche la policy iptables in esecuzione sulla macchina per determinare quali regole Snort sono applicabili.

È possibile scaricare il codice sorgente dal sito ufficiale.

Per estrarre ed eseguire il file a partire dal pacchetto bz2:

\begin{verbatim}
tar xfj fwsnort-1.0.tar.bz2
cd /usr/local/src/fwsnort-1.0 
./install.pl
\end{verbatim}

Il file di configurazione si trova nella directory \texttt{/etc/fwsnort/fwsnort.conf}

Nel file di configurazione vengono definite delle variabili che corrispondono a indirizzi IP o a porte e vengono utilizzate nelle regole.

\begin{verbatim}
HOME_NET                10.222.111.0/24;
EXTERNAL_NET            !$HOME_NET;

HTTP_SERVERS            $HOME_NET;
DNS_SERVERS             $HOME_NET;

SSH_PORTS               [64022,65022];
HTTP_PORTS              80;
SHELLCODE_PORTS         !80;
\end{verbatim}

Si noti che è stato scelto un range di porte SSH al posto della porta di default 22. Questo per motivi di sicurezza: leggendo i log SSH nella directory \texttt{/var/log/secure} e i log registrati dallo stesso Snort, si è visto che esistono degli script automatizzati che tentano continuamente di instaurare una connessione SSH con il server. Per eliminare il problema, SSH è stato reso disponibile su un altre porte.

Un esempio di regola facente parte delle community rules che usa le variabili specificate durante la configurazione:

\begin{verbatim}
alert tcp $EXTERNAL_NET any -> $HTTP_SERVERS $HTTP_PORTS 
(msg:"SERVER-APACHE Apache Tomcat view source attempt";
 flow:to_server,established; content:"%252ejsp"; http_uri;
metadata:ruleset community, service http; 
reference:bugtraq,2527; reference:cve,2001-0590;
classtype:web-application-attack; sid:1056; rev:16;)
\end{verbatim}

~\cite{FWSnort} ~\cite{techrepublic}.