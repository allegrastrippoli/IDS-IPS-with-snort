I dati contenuti in un computer permettono di raggiungere un alto grado di conoscenza circa il contesto in cui esso viene utilizzato. In un computer privato potrebbero essere memorizzati dati di carte di credito, numeri di conti correnti bancari, fotografie, file personali. In un computer utilizzato da un'azienda potrebbero essere memorizzati i dati dei clienti, rendiconti finanziari, documenti etc... Nel momento in cui un computer entra a far parte di una rete ed è in grado di comunicare con altri dispositivi collegati direttamente o raggiungibili indirettamente, questi dati vanno protetti affinché non vadano persi, non vengano compromessi o finiscano nelle mani sbagliate. La rapida proliferazione dei dati digitali rende il problema della sicurezza e della protezione dei dati così comune da riguardare chiunque utilizzi computer, smartphone, tablet ...

Nello specifico ci si cala nel contesto di un'azienda come tante, che ha bisogno di esporre dei servizi su internet e teme per la sicurezza della propria rete. La rete nel suo stato primordiale ha due componenti: un router e una macchina virtuale su cui è installato un firewall, un primo strumento di difesa che segue delle politiche di sicurezza e filtra il traffico applicando delle regole.
Gli obiettivi da perseguire sono due: è necessario rivedere la morfologia della rete per far si che questa possa ospitare dei web server e file server, ossia possa offrire dei servizi su internet. È necessario aggiungere un altro strato difensivo (oltre al firewall) che renda la rete più sicura.
La strategia è quella di introdurre una seconda macchina virtuale che ospiti i servizi richiesti dall'azienda. Le due macchine svolgono ruoli distinti: una si occupa di gestire gli aspetti legati alla sicurezza, l'altra dei servizi.
Dividere la rete in moduli, ossia tenere separate aree che svolgono funzioni diverse è una buona norma, il vantaggio è quello di poter correggere, gestire i guasti senza propagare le modifiche per tutta la rete.
La macchina del firewall viene frapposta tra il router e la macchina (o le macchine) dei servizi. In questo modo può filtrare tutto il traffico che proviene da internet ed è diretto all' (o agli) host.
Se l'azienda con il tempo decidesse di espandersi e di aggiungere altre macchine, questa configurazione permette di farlo. A valle della macchina che ospita il firewall possono trovarsi tanti host quanti sono gli indirizzi IP disponibili nella LAN. Per questo la rete si dice scalabile.

In questa configurazione la responsabilità di difendere la rete è affidata completamente al firewall. L'impostazione del firewall è compito dell'amministratore di rete, che dovrà progettarlo in modo che siano consentite solo ed esclusivamente le connessioni necessarie al lavoro dell'azienda. Un firewall configurato male, o non configurato proprio, lascerebbe passare traffico indesiderato non svolgendo la sua funzione di difesa. Inoltre non è uno strumento capace di adattarsi autonomamente nel caso in cui si verificassero attacchi.
Per risolvere queste problematiche si decide di realizzare un sistema di rilevazione e prevenzione delle intrusioni con un software open source chiamato Snort. I vantaggi sono molteplici: Snort permette di monitorare il traffico, catturare e analizzare pacchetti, sospendere possibili attività sospette, prevenire attacchi. Esso basa la propria strategia di rilevazione su una conoscenza profonda dei protocolli di rete e delle vulnerabilità dei servizi. Un sistema di difesa di questo tipo, che permette di riconoscere attività anomale e di agire per isolarle, eleva gli standard di sicurezza della rete.

Nel primo capitolo vengono elencate alcune delle minacce più comuni in rete, dal phishing agli attacchi DoS e DDoS, vengono poi introdotti alcuni strumenti di difesa come il firewall e i sistemi di rilevazione e prevenzione delle intrusioni (IDS e IPS).

Nel secondo capitolo viene presentato Snort un IDS/IPS network-based. Dopo averne dato la definizione, vengono presentate le varie modalità di esecuzione al fine di esprimerne le potenzialità. Viene posta attenzione sulla strategia di detection rule-based (basata su regole) utilizzata da Snort. Le regole sono uno strumento raffinato e versatile per distinguere i pacchetti autorizzati a transitare da quelli che invece vanno bloccati. Si noti che si tratta di uno strumento di rilevazione e non di prevenzione. Per prevenire le intrusioni è necessario fare in modo che, nel momento in cui Snort rileva un comportamento sospetto, il firewall sia in grado di respingere l'indirizzo IP sorgente del traffico indesiderato. Alla fine del capitolo vengono accennati vari strumenti di appoggio utilizzati da Snort per la registrazione delle informazioni relative ai pacchetti.

Nel terzo capitolo vengono presentati i requisiti imposti dall'azienda. I vincoli, gli obiettivi, le linee guida da seguire.

Nel quarto capitolo viene presentato il progetto. Gli obiettivi del progetto sono: modificare la struttura della rete rispetto al modello proposto dall'azienda e introdurre il sistema di rilevazione e prevenzione delle intrusioni con Snort.

Nel quinto capitolo si procede con l'installazione e la configurazione di Snort in modo che questo si adegui alle esigenze della rete. Vengono presentati anche PulledPork e FwSnort, il primo è uno script che permette di mantenere aggiornate le regole di Snort, il secondo invece permette la traduzione di regole.

Nel sesto capitolo vengono messe in atto tre verifiche funzionali. Il primo test consiste nel generare del traffico sulle porte di interesse del server e verificare che Snort lo rilevi. Il secondo consiste nell'eseguire uno scan delle porte, e anche in questo caso verificare che Snort lo notifichi. L'ultimo test vede il server vittima di un attacco DoS. Snort deve essere in grado di rilevare e prevenire l'attacco.